\documentclass[aspectratio=169]{beamer}

% beamer config
\usetheme{Berkeley}
% \AtBeginSection{\frame{\sectionpage}}

% package imports
\usepackage{graphicx}
\graphicspath{{./figs/}}

% Macros
\newcommand{\todo}[1]{\textcolor{red}{\textbf{[#1]}}}

\title[PhotoHunter]{PhotoHunter: A Citizen-Scientist Game with User-in-the-loop
  Data Confirmation for Collecting Computer Vision Datasets of
  Geo-tagged Imagery through Crowd-Sourcing Data Collection;
  Abstracting Research using an Interactive, Game-based Mobile
  Application for Large Dataset Creation}

\author[]{Connor Greenwell \and Ryan Baltenberger
  \and J.\ David Smith \and Aaron Bradshaw}

\institute{QuesoTech.com}

\begin{document}

\maketitle

\section{Computer Vision}

\begin{frame}{Computer Vision}
  \begin{itemize}

    \item Artificial Intelligence - the goal of creating intelligence for
          machines/software

    \item Computer Vision - teaching computers to understand the natural world
          through images

    \item Data to train or test on is integral to success

    \item Collecting and labeling data is time consuming

  \end{itemize}
\end{frame}

\section{Citizen-Science}

\begin{frame}{Citizen-Science}
  \begin{itemize}

    \item Citizen-Science - form of crowd-sourcing where participants
          collect/analyze data

    \item Typically performed tasks are simple (\textit{e.g.} take a photo,
          measure temperature, translate text, etc.)

    \item Humans can perform simple tasks easily and in large quantities,
          often with little time or effort

    \item Collecting large amounts of data is intractable for a single
          researcher

  \end{itemize}
\end{frame}

\section{Why a game?}

\begin{frame}{Why a Game?}
  \begin{itemize}

    \item Contributions to citizen-science style datasets drop off after
          the initial interest wears off

    \item Using ``gamification'' can mitigate this decline

    \item Our goal - abstract out the data collection, tell users they
          are competing in a global scavenger hunt

    \item Users will stay engaged/contributing to earn more points

    \item PhotoHunter will assign points for collecting and reviewing data

    \item Leader-boards will encourage competition between users

  \end{itemize}
\end{frame}

\section{Overview}

\begin{frame}{Overview}
  \centering
  \includegraphics[width=\textwidth,height=\textheight,keepaspectratio]{ss_flowchart}
\end{frame}

\section{Backend}

\begin{frame}{Backend}
  \begin{itemize}
    \item Research groups may be created on request.

    \item Researcher interface where they may request datasets.

    \item define GPS bounding boxes, time frames, and other metadata they are
      interested in for a given dataset.
  \end{itemize}
\end{frame}

\begin{frame}{ER Diagram}
  \centering
  \includegraphics[width=\textwidth,height=\textheight,keepaspectratio]{er}
\end{frame}

\begin{frame}{Researchers Interface}
  \centering
  \includegraphics[width=\textwidth,height=\textheight,keepaspectratio]{researchers}
\end{frame}

\section{PhotoHunter}

\begin{frame}{PhotoHunter}
  \begin{itemize}

    \item Mobile application

    \item Users presented with list of categories

    \item Each category represents a dataset in their location

    \item Gain points for completed tasks

    \item Data collection only occurs within this mobile app so
          we can take advantage of built in sensors

  \end{itemize}
\end{frame}

\begin{frame}{PhotoHunter}
  \begin{columns}[c]
    \begin{column}{0.5\columnwidth}
      \centering
      \includegraphics[width=\textwidth,height=\textheight,keepaspectratio]{ss_photohunter_dataset}
    \end{column}
    \begin{column}{0.5\columnwidth}
      \centering
      \includegraphics[width=\textwidth,height=\textheight,keepaspectratio]{ss_photohunter_upload}
    \end{column}
  \end{columns}
\end{frame}

\section{QuickPic}

\begin{frame}{QuickPic}
  \begin{columns}[c]
    \begin{column}{0.5\columnwidth}
      \centering
      \includegraphics[width=\textwidth,height=\textheight,keepaspectratio]{ss_quickpic_image}
    \end{column}
    \begin{column}{0.5\columnwidth}
      \centering
      \includegraphics[width=\textwidth,height=\textheight,keepaspectratio]{ss_quickpic_options}
    \end{column}
  \end{columns}
\end{frame}

\begin{frame}{QuickPic}
  \begin{itemize}
    \item Accuracy determined on the basis of other user's label choices at time
      of submission.

    \item Gain points for labeling images, depending on how accurate their
      choice is.

    \item Each image shown to numerous users. By looking at the distribution of
      choices, we can automatically determine the accuracy of the label.

    \item This accuracy information will be passed along to researchers.

  \end{itemize}
\end{frame}

\frame{\centering Thanks!}

\end{document}
