\documentclass{article}

\usepackage[scale=0.7]{geometry}
\usepackage{hyperref, tablefootnote, color, graphicx, float}
\newcommand{\todo}[1]{\textcolor{red}{\textbf{[#1]}}}

\title{PhotoHunter: A Citizen-Scientist Game with User-in-the-loop
  Data Confirmation for Collecting Computer Vision Datasets of
  Geo-tagged Imagery through Crowd-Sourcing Data Collection; Abstracting
  Research using an Interactive, Game-based Mobile Application for Large
  Dataset Creation
  \footnote{Project location:
    \url{https://github.com/QuesoTech/PhotoHunter}
  } \\
  \textit{CS 499 - Senior Design, University of Kentucky}
}

\author{Connor Greenwell \and Ryan Baltenberger
  \and J.\ David Smith \and Aaron Bradshaw
\and Scott Workman\footnote{Customer}}

\begin{document}
\maketitle

\begin{abstract}
  The PhotoHunter project is a system intended to simplify the creation of
  datasets for computer vision research by the process of "gamification". The
  system is composed of two mobile applications and a web interface. The web
  interface allows researchers to specify a set of images based on their needs.
  Based on these specifications, the first mobile application generates a list of
  descriptions for photos. Users then capture photos matching these descriptions
  and submit them in a scavenger-hunt style game. The second mobile application
  uses the submitted photos to quiz users. The players are briefly shown the
  images and then asked to label them. Based on the most-commonly provided
  answer, the correct label for the image can be determined. The labelled images
  are then provided to the researchers that requested the dataset. The system
  provides entertainment for users and useful datasets for researchers.
\end{abstract}

\section{Disclaimer}
This project has been designed and implemented as a part of the requirements
for CS-499 Senior Design Project for Spring 2014 semester.  While the authors
make every effort to deliver a high quality product, we do not guarantee that
our products are free from defects.  Our software is provided "as is," and you
use the software at your own risk.

We make no warranties as to performance, merchantability, fitness for a
particular purpose, or any other warranties whether expressed or implied.

No oral or written communication from or information provided by the authors or
the University of Kentucky shall create a warranty.

Under no circumstances shall the authors or the University of Kentucky be
liable for direct, indirect, special, incidental, or consequential damages
resulting from the use, misuse, or inability to use this software, even if the
authors or the University of Kentucky have been advised of the possibility of
such damages.

\section{Product Specifications}
Specs for the project go here.

\section{Product Planning}
Planning on the project goes here.

\section{Schedule and Milestones}
Some stuff about our schedule goes here.

\section{Platforms, Tools, and Languages}
Stuff about what we used goes here.

\section{Design}
Port over some stuff about the design

\section{Implementation}
Talk about how we implemented everything.

\section{Future Enhancements/Maintenance}
How to make it better.

\section{Conclusions}
Wrap it up.

\section{References}
Cite everything we used.

\end{document}
