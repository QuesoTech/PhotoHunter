\documentclass{article}

\usepackage[scale=0.7]{geometry}
\usepackage{hyperref, tablefootnote, color, graphicx, float}
\usepackage{titlesec}

\setcounter{secnumdepth}{4}

\titleformat{\paragraph}
{\normalfont\normalsize\bfseries}{\theparagraph}{1em}{}
\titlespacing*{\paragraph}
{0pt}{3.25ex plus 1ex minus .2ex}{1.5ex plus .2ex}

\newcommand{\todo}[1]{\textcolor{red}{\textbf{[#1]}}}

\title{PhotoHunter: A Citizen-Scientist Game with User-in-the-loop
  Data Confirmation for Collecting Computer Vision Datasets of
  Geo-tagged Imagery through Crowd-Sourcing Data Collection; Abstracting
  Research using an Interactive, Game-based Mobile Application for Large
  Dataset Creation
  \footnote{Project location:
    \url{https://github.com/QuesoTech/PhotoHunter}
  } \\
  \textit{CS 499 - Senior Design, University of Kentucky}
}

\author{Connor Greenwell \and Ryan Baltenberger
  \and J.\ David Smith \and Aaron Bradshaw
\and Scott Workman\footnote{Customer}}

\begin{document}
\maketitle

\begin{abstract}
  The PhotoHunter project is a system intended to simplify the creation of
  datasets for computer vision research by the process of "gamification". The
  system is composed of two mobile applications and a web interface. The web
  interface allows researchers to specify a set of images based on their needs.
  Based on these specifications, the first mobile application generates a list of
  descriptions for photos. Users then capture photos matching these descriptions
  and submit them in a scavenger-hunt style game. The second mobile application
  uses the submitted photos to quiz users. The players are briefly shown the
  images and then asked to label them. Based on the most-commonly provided
  answer, the correct label for the image can be determined. The labelled images
  are then provided to the researchers that requested the dataset. The system
  provides entertainment for users and useful datasets for researchers.
\end{abstract}




\section{Disclaimer}
This project has been designed and implemented as a part of the requirements
for CS-499 Senior Design Project for Spring 2014 semester.  While the authors
make every effort to deliver a high quality product, we do not guarantee that
our products are free from defects.  Our software is provided "as is," and you
use the software at your own risk.

We make no warranties as to performance, merchantability, fitness for a
particular purpose, or any other warranties whether expressed or implied.

No oral or written communication from or information provided by the authors or
the University of Kentucky shall create a warranty.

Under no circumstances shall the authors or the University of Kentucky be
liable for direct, indirect, special, incidental, or consequential damages
resulting from the use, misuse, or inability to use this software, even if the
authors or the University of Kentucky have been advised of the possibility of
such damages.


\section{Introduction}
Intro goes here.

\section{Product Specifications}

\subsection{Environment}

\subsubsection{Web Browsers}
The web interface for the PhotoHunter Project will be compatible with the
versions of Google Chrome and Mozilla Firefox that are available on April 1,
2015. This interface will also employ a responsive design to accommodate users
visiting on a mobile device's browser.

\subsubsection{Mobile Application}
The two mobile applications for the PhotoHunter Project will be developed with
cross-platform tool, allowing the applications to be deployed to Android
version 4.4.2 (Kit-Kat).

\subsection{Architecture}

\subsubsection{Overview}
The components of the PhotoHunter Project will each utilize the same database
for different purposes.

\subsubsection{Web Interface}
The web interface will allow researchers to specify an image dataset based on
their needs. This specification is then used as a topic for the PhotoHunter
mobile application. Researchers may also use the web interface retrieve their
finalized dataset after the images have been labelled by the QuickPic mobile
application.

\subsubsection{PhotoHunter}
This mobile application generates lists based on the topics provided by
researchers through the web interface. Users playing the PhotoHunter game
capture photos matching the topics in the list. These photos are uploaded to
the overall system's database for use in dataset generation.

\subsubsection{QuickPic}
This mobile application uses the photos and topics provided by the PhotoHunter
application and web interface to quiz users. Users are briefly shown one of the
photos. Afterwards, the users are asked to label the photo based on a set of
choices. The answer provided by the user is sent to the central system.

\subsubsection{Backend}
The backend of the system will generate lists for the PhotoHunter application.
The backend will also analyze the data provided by QuickPic to predict the
correctness of user-provided labels. Finally, the backend will compile and
generate datasets for researchers to download.

\subsection{Features}

\subsubsection{Web Interface}
The web interface will provide researchers with tools for dataset
specification, tracking, and retrieval.

  \paragraph{Researcher Accounts}
  The web interface will allow researchers to create accounts to login. An
  account will be necessary to use the other features of the web interface.

  \paragraph{Dataset Requests}
  When logged in to the system, users can create a new request for a dataset.
  This request will be a description of the data needed and the desired size of
  the dataset.

  \paragraph{Dataset Status}
  The system will allow users to view the current status of their dataset based
  on the current size. This status will provide information regarding recent
  activity in the dataset and time since the dataset was initialized.

\subsubsection{PhotoHunter Application}
The PhotoHunter mobile application provides a scavenger-hunt experience where
users are given lists of photo descriptions. Users then capture and submit
photos matching these descriptions. Based on the number and quality of
submissions, users are ranked against one another, introducing a competitive
element to the process.

  \paragraph{User Accounts}
  The application will allow users to log in through Facebook. An account will
  be necessary for the user to submit photos.

  \paragraph{Hunt Lists}
  The application will provide lists of topics to the user.

  \paragraph{Photo Submission}
  The application will allow users to take photos with their on-device camera.
  These photos can then be submitted to one of the options on the user's current
  list of topics.

\subsubsection{Quick Pic}
The Quick Pic mobile application is quiz game where users quickly identify
images based on a provided set of labels. Users are briefly shown an image.
Then users must select the most correct label from a set of choices. Based on
their correctness and speed, users are given points and ranked against one
another.

  \paragraph{User Accounts}
  The application requires users to log in through Facebook.

  \paragraph{Statistics}
  The application will allow users to view quiz statistics, such as percentage
  correct, average response time, and total points.

  \paragraph{Image Flash}
  The application will briefly show the user an image pulled from the
  PhotoHunter Project's database.

  \paragraph{Labels}
  After showing a user an image, the application will provide the user with a
  list of labels. These labels will have varying degrees of relevancy. One label
  will be determined by the topic in which the photo was submitted as in the
  PhotoHunter application.

\subsubsection{Backend}
The backend of the PhotoHunter Project will manage the data between the different
applications. The backend will use the data to generate databases for researchers
using the PhotoHunter system from the web.

  \paragraph{Database Management}
  The backend will take dataset requests from the web interface and update the
  image database accordingly.

  \paragraph{Photo Retrieval}
  The backend will receive images and metadata from the PhotoHunter application
  and store them in a database.

  \paragraph{Photo Posting}
  The backend will provide images to the Quick Pic application from the database.

  \paragraph{Photo Selection}
  The backend will choose photos to provide to Quick Pic based on the current
  information available for the photos, ensuring that sufficient data is collected
  for each image.

  \paragraph{Photo Processing}
  Based on the answers provided in the Quick Pic application, the backend will
  predict the correct label of the images in the database.

  \paragraph{Dataset Creation}
  The backend will monitor the states of the dataset requests and the database
  and create the datasets once the requirements have been met.

\subsection{Interfaces}
The PhotoHunter Project will have three central interfaces: two mobile
applications and a web interface.

\subsubsection{Web Interface}
The web interface component will have a login menu. Once logged in the, user
will be presented with a control panel. This control panel will contain various
options including:

\begin{enumerate}

  \item \textbf{Dataset Request:} A form for requesting a new dataset.

  \item \textbf{Dataset Status:} An information panel for viewing statistics
        about any datasets in development.

\end{enumerate}

\subsubsection{PhotoHunter}
The PhotoHunter mobile application will have a main login screen upon start.
After logging in, three main menu options are available, including:

\begin{enumerate}

  \item \textbf{Photo Hunt:} Provides the user with a list of topics. On this
        view, the user may also choose to take a photo. After taking a photo, the user
        may choose a relevant category under which to submit the photo.

  \item \textbf{Statistics:} Allows the user to view personal statistics.

\end{enumerate}

\subsubsection{QuickPic}
The QuickPic mobile application will have a main login screen upon start. The
following main menu will have the following options:

\begin{enumerate}

  \item \textbf {Photo Quiz:} Quizzes the user by showing them photos and
        asking for the correct label.

  \item \textbf{Statistics:} Allows the user to view personal statistics.

\end{enumerate}

\subsection{Installation}
The web interface component of the PhotoHunter Project will require no
installation. The user will login to the site using any standard internet
browser. The mobile applications will need to be downloaded from the Google
Play Store once accepted and deployed.


\section{Product Planning}
Planning on the project goes here.

\section{Schedule and Milestones}
Some stuff about our schedule goes here.

\section{Platforms, Tools, and Languages}

We used a variety of languages and tools to develop and manage our project. The
choices we made were motivated primarily by the tasks we expected to encounter
in our project.

\subsection{Backend}

The backend is built in the Go language. This language was chosen because it
has good support for concurrency, which is important for a Web API. It also has
excellent libraries available for API development.

The backend stores data in a PostgreSQL database that has the PostGIS extension
installed. PostgreSQL was chosen because of the availability of the PostGIS
extension, which enables accurately and easily storing GIS (Geographic
Information System) data in the database.

The standard Go tools included in the distribution are used to build and run
the backend.

\subsection{Web Interface}

The web interface is built in the Go language using the Gorilla set of
libraries. It is implemented in HTML and CSS and includes the
Bootstrap CSS theme.

The standard Go tools included in the distribution are used to build and run
the web interface.

\subsection{PhotoHunter}

The PhotoHunter application is built using the Cordova framework,
which allows deployment of a HTML/JS/CSS web application to multiple mobile
platforms. The application was written primarily in JavaScript, with HTML and
CSS tying it together. It uses the Bootstrap CSS theme. The Cordova build tool
is used to generate application binaries.

The application is only confirmed to work in Android 4.4.2 at present. Our use
of the Cordova framework should make porting to later Android versions, iOS,
and Windows Phone require little effort. However, we have not tested on
anything aside from Android 4.4.2.

\subsection{QuickPic}

The QuickPic application is built using the Cordova framework,
which allows deployment of a HTML/JS/CSS web application to multiple mobile
platforms. The application was written primarily in JavaScript, with HTML and
CSS tying it together. It uses the Bootstrap CSS theme. The Cordova build tool is used to generate application
binaries.

The application is written in Node.js-style (multiple .js files, nicely
compartmentalized). The Gulp tool is used to compile the JavaScript
for this application into a single file for deployment.

The application is only confirmed to work in Android 4.4.2 at present. Our use
of the Cordova framework should make porting to later Android versions, iOS,
and Windows Phone require little effort. However, we have not tested on
anything aside from Android 4.4.2.

\section{Design}
Port over some stuff about the design

\section{Implementation}
Talk about how we implemented everything.

\section{Future Enhancements/Maintenance}
How to make it better.

\section{Conclusions}
Wrap it up.

\section{References}
Cite everything we used.

\end{document}
