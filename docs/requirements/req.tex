\documentclass{article}

\usepackage{hyperref, tablefootnote, color}

\newcommand{\todo}[1]{\textcolor{red}{\textbf{[#1]}}}

\title{PhotoHunter Requirements Specification
}

\author{Connor Greenwell \and Ryan Baltenberger 
  \and J.\ David Smith \and Aaron Bradshaw
  \and Scott Workman\footnote{Advisor}}

\begin{document}

\maketitle

\section{Introduction}
\subsection{Project}
The PhotoHunter system provides a simplified means of creating datasets for
computer vision. The system consists of two mobile applications an a web
interface. 

\subsection{Purpose}
The purpose of this document is to clearly define the expected functionality of
the "PhotoHunter" system. This document is to serve as a reference and
guideline for the client and the developers of the project. The document will:
1) provide a complete overview of the system and its subcomponents; 2) discuss
constraints and the scope of the project; 3) discuss the purpose of the system. 

\subsection{Definitions, Acronyms, and Abbreviations}

\begin{enumerate}

  \item \textbf{PhotoHunter Project}: The overall system, composed of PhotoHunter,
        Quick Pic, and a web interface.
  \item \textbf{PhotoHunter}: The mobile application which provides users
        descriptions of photos to capture in the style of a scavenger-hunt. 
  \item \textbf{Quick Pic}: The mobile application which quizzes users, asking
        them to label images that they are briefly shown.
  \item \textbf{Gamification}: Using game mechanics to solve problems.

\end{enumerate}

\section{Overview}
\subsection{Overall Description}
The PhotoHunter project is a system intended to simplify the creation of
datasets for computer vision research by the process of "gamification". The
system is composed of two mobile applications and a web interface. The web
interface allows researchers to specify a set of images based on their needs.
Based on these specifications, the first mobile application generates a list of
descriptions for photos. Users then capture photos matching these descriptions
and submit them in a scavenger-hunt style game. The second mobile application
uses the submitted photos to quiz users. The players are briefly shown the
images and then asked to label them. Based on the most-commonly provided
answer, the correct label for the image can be determined. The labelled images
are then provided to the researchers that requested the dataset. The system
provides entertainment for users and useful datasets for researchers.

\subsection{Document Structure}
The remaining sections of the document discuss the specific requirements,
constraints, and risks of the project. These sections are:
\begin{enumerate}

  \item \textbf{Environment}: A description of the hardware and software
        environments in which the PhotoHunter project will be deployed.
  \item \textbf{Architecture}: A high level description of the overall
        architecture of the PhotoHunter project.
  \item \textbf{Features}: Describes all of the features of the PhotoHunter
        project.
  \item \textbf{Interfaces}: Describes the methods in which users interact with
        the PhotoHunter project.
  \item \textbf{Installation}: Description of the installation process required
        for PhotoHunter.
  \item \textbf{Constraints}: Discusses constraints on the PhotoHunter system.
  
  \item \textbf{Risks}: A discussion of the possible risks involved in designing
        and building the PhotoHunter project.

\end{enumerate}

\section{Environment}

\subsection{Web Browsers}
The web interface for the PhotoHunter Project will be compatible with the
versions of Google Chrome and Mozilla Firefox that are available on
April 1, 2015. This interface will also employ a
responsive design to accommodate users visiting on a mobile device's browser.

\subsection{Mobile Application}
The two mobile applications for the PhotoHunter Project will be developed with
cross-platform tool, allowing the applications to be deployed to Android
version 4.4.2 (Kit-Kat) and, if time permits, iOS version 8.

\section{Architecture}

\subsection{Overview}
The components of the PhotoHunter Project will each utilize the same database
for different purposes.

\subsection{Web Interface}
The web interface will allow researchers to specify an image dataset based on
their needs. This specification is then used as a topic for the PhotoHunter
mobile application. Researchers may also use the web interface retrieve their
finalized dataset after the images have been labelled by the Quick Pic mobile
application. 

\subsection{PhotoHunter}
This mobile application generates lists based on the topics provided by
researchers through the web interface. Users playing the PhotoHunter game
capture photos matching the topics in the list. These photos are uploaded to
the overall system's database for use in dataset generation.

\subsection{Quick Pic}
This mobile application uses the photos and topics provided by the PhotoHunter
application and web interface to quiz users. Users are briefly shown one of the
photos. Afterwards, the users are asked to label the photo based on a set of
choices. The answer provided by the user is sent to the central system. 

\subsection{Backend}
The backend of the system will generate lists for the PhotoHunter application.
The backend will also analyze the data provided by Quick Pic to predict the
correctness of user-provided labels. Finally, the backend will compile and
generate datasets for researchers to download.

\section{Features}

\subsection{Web Interface}
The web interface will provide researchers with tools for dataset
specification, tracking, and retrieval. 

	\subsubsection{Researcher Accounts}
  The web interface will allow researchers to create accounts to login. An
  account will be necessary to use the other features of the web interface.
	
	\subsubsection{Dataset Requests}
  When logged in to the system, users can create a new request for a dataset.
  This request will be a description of the data needed and the desired size of
  the dataset.
	
	\subsubsection{Dataset Status}
  The system will allow users to view the current status of their dataset based
  on the current size. This status will provide information regarding recent
  activity in the dataset and time since the dataset was initialized. 
	
	\subsubsection{Dataset Download}
  The system will allow users to download their dataset after it is complete.
  Additionally, the system will allow users to download incomplete datasets. 
	
	
\subsection{PhotoHunter Application}
The PhotoHunter mobile application provides a scavenger-hunt experience where
users are given lists of photo descriptions. Users then capture and submit
photos matching these descriptions. Based on the number and quality of
submissions, users are ranked against one another, introducing a competitive
element to the process.

	\subsubsection{User Accounts}
  The application will allow users to create profiles to login. An account will
  be necessary for the user to submit photos.
	
	\subsubsection{Statistics}
	The application will allow users to track their statistics when logged in.
	
	\subsubsection{Hunt Lists}
	The application will provide lists of topics to the user.
	
	\subsubsection{Photo Submission}
  The application will allow users to take photos with their on-device camera.
  These photos can then be submitted to one of the options on the user's current
  list of topics.
	
	\subsubsection{Leaderboards}
  The application will allow users to view their current ranking against all
  other users. This ranking will be determined by the number of submissions a
  user has made.
	
\subsection{Quick Pic}
The Quick Pic mobile application is quiz game where users quickly identify
images based on a provided set of labels. Users are briefly shown an image.
Then users must select the most correct label from a set of choices. Based on
their correctness and speed, users are given points and ranked against one
another.

	\subsubsection{User Accounts}
	The application requires users to create profiles. 
	
	\subsubsection{Statistics}
  The application will allow users to view quiz statistics, such as percentage
  correct, average response time, and total points.
	
	\subsubsection{Image Flash}
  The application will briefly show the user an image pulled from the
  PhotoHunter Project's database.
	
	\subsubsection{Labels}
  After showing a user an image, the application will provide the user with a
  list of labels. These labels will have varying degrees of relevancy. One label
  will be determined by the topic in which the photo was submitted as in the
  PhotoHunter application. 
	
	\subsubsection{Leaderboards}
  The application will allow users to view their ranking against all other
  users based on their cumulative points.
	
\subsection{Backend}
The backend of the PhotoHunter Project will manage the data between the different
applications. The backend will use the data to generate databases for researchers
using the PhotoHunter system from the web.

        \subsubsection{Database Management}
        The backend will take dataset requests from the web interface and update the
        image database accordingly.

        \subsubsection{Photo Retrieval}
        The backend will receive images and metadata from the PhotoHunter application 
        and store them in a database.

        \subsubsection{Photo Posting}
        The backend will provide images to the Quick Pic application from the database.

        \subsubsection{Photo Selection}
        The backend will choose photos to provide to Quick Pic based on the current 
        information available for the photos, ensuring that sufficient data is collected
        for each image.

        \subsubsection{Photo Processing}
        Based on the answers provided in the Quick Pic application, the backend will 
        predict the correct label of the images in the database.

        \subsubsection{Dataset Creation}
        The backend will monitor the states of the dataset requests and the database 
        and create the datasets once the requirements have been met. 


\section{Interfaces}
The PhotoHunter Project will have three central interfaces: two mobile
applications and a web interface.

\subsection{Web Interface}
The web interface component will have a login menu. Once logged in the, user
will be presented with a control panel. This control panel will contain various
options including:

\begin{enumerate}

  \item \textbf{Dataset Request:} A form for requesting a new dataset.
  
  \item \textbf{Dataset Status:} An information panel for viewing statistics
        about any datasets in development.
  
  \item \textbf{Dataset Download:} A tool allowing users the download any
        complete or partially complete datasets.

\end{enumerate}
	
\subsection{PhotoHunter}
The PhotoHunter mobile application will have a main login screen upon start.
After logging in, three main menu options are available, including:

\begin{enumerate}

  \item \textbf{Photo Hunt:} Provides the user with a list of topics. On this
        view, the user may also choose to take a photo. After taking a photo, the user
        may choose a relevant category under which to submit the photo.

  \item \textbf{Statistics:} Allows the user to view personal statistics.

  \item \textbf{Leaderboard:} Allows the user to view global leaderboards.

\end{enumerate}
	
\subsection{Quick Pic}
The Quick Pic mobile application will have a main login screen upon start. The
following main menu will have the following options:

\begin{enumerate}

  \item \textbf {Photo Quiz:} Quizzes the user by showing them photos and
        asking for the correct label.

  \item \textbf{Statistics:} Allows the user to view personal statistics.

  \item \textbf{Leaderboard:} Allows the user to view global leaderboards.

\end{enumerate}
	 
\section{Installation}
The web interface component of the PhotoHunter Project will require no
installation. The user will login to the site using any standard internet
browser. The mobile applications will need to be downloaded from the Google
Play App Store or the iTunes App Store, depending on the user's device.

\section{Constraints/Risks}

\subsection{Time}
The PhotoHunter Project should be completed by the week of April 27th, 2015.

\subsection{Memory}
The PhotoHunter Project should be scalable enough to handle and process
potentially thousands of images. There will be enough memory on the server to
store these images and information about them.

\end{document}
