\documentclass{article}

\usepackage{hyperref, tablefootnote, color, graphicx, float}
\graphicspath{{../midterm_presentation/figs/}}
\newcommand{\todo}[1]{\textcolor{red}{\textbf{[#1]}}}

\title{PhotoHunter Testing Plan Documentation
}

\author{Connor Greenwell \and Ryan Baltenberger
  \and J.\ David Smith \and Aaron Bradshaw
\and Scott Workman\footnote{Advisor}}

\begin{document}

\maketitle

\section{Overview}
The PhotoHunter project is a system intended to simplify the creation of
datasets for computer vision research by the process of "gamification". The
system is composed of two mobile applications and a web interface. The web
interface allows researchers to specify a set of images based on their needs.
Based on these specifications, the first mobile application generates a list of
descriptions for photos. Users then capture photos matching these descriptions
and submit them in a scavenger-hunt style game. The second mobile application
uses the submitted photos to quiz users. The players are briefly shown the
images and then asked to label them. Based on the most-commonly provided
answer, the correct label for the image can be determined. The labelled images
are then provided to the researchers that requested the dataset. The system
provides entertainment for users and useful datasets for researchers.

\section{Unit Testing/Functional Testing}
\subsection{PhotoHunter}

For the Photohunter app, we will test each function on the Android (4.4.2)
mobile platform.  The tests we will perform are:
\begin{itemize}

  \item Logging into the app through Facebook and create an account
 
  \item Logging into the app with an already existing account

  \item Retrieving the correct device location

  \item Sending the current location to the Photohunter API and receiving
        a list of datasets

  \item Selecting a dataset from the generated list
  
  \item Taking a picture with the built in camera and uploading to the 
        database through the Photohunter API

  \item Correctly populating the leaderboard with point data

\end{itemize}

\subsection{Quick Pic}
\subsection{Researcher Interface}
 

\section{System Testing}
System testing will focus on the interaction between the PhotoHunter app, Quick Pic app, researcher interface, and the PhotoHunter API. 
The expected flow of the project is as follows:
\begin{itemize}
\item The researcher interface should allow users to define and describe new datasets.
\item The PhotoHunter application should successfully insert new images with relevant information into the database.
\item The PhotoHunter API should serve images and relevant tags from the database to the Quick Pic application.
\item The Quick Pic application should successfully provide user feedback to the images provided by the API. 
\item The PhotoHunter API should analyze and package the data provided by the PhotoHunter and Quick Pic apps, and provide a download to the researcher interface.
\item The researcher interface should allow users to download the package generated by the PhotoHunter API.
\end{itemize}

The system will be tested on the latest stable versions of Google Chrome and Android available as of April, 2015. 

\section{Customer Testing}
The customer will be given access to the researcher interface and the packages for the Android applications. The intention is to provide the user with these builds early enough to correct any serious issue before the project deadline. 
\end{document}
