\documentclass{article}

\usepackage{hyperref, tablefootnote, color, graphicx, float}
\graphicspath{{../midterm_presentation/figs/}}
\newcommand{\todo}[1]{\textcolor{red}{\textbf{[#1]}}}

\title{PhotoHunter Testing Plan Documentation
}

\author{Connor Greenwell \and Ryan Baltenberger
  \and J.\ David Smith \and Aaron Bradshaw
\and Scott Workman\footnote{Advisor}}

\begin{document}

\maketitle

\section{Overview}
The PhotoHunter project is a system intended to simplify the creation of
datasets for computer vision research by the process of "gamification". The
system is composed of two mobile applications and a web interface. The web
interface allows researchers to specify a set of images based on their needs.
Based on these specifications, the first mobile application generates a list of
descriptions for photos. Users then capture photos matching these descriptions
and submit them in a scavenger-hunt style game. The second mobile application
uses the submitted photos to quiz users. The players are briefly shown the
images and then asked to label them. Based on the most-commonly provided
answer, the correct label for the image can be determined. The labelled images
are then provided to the researchers that requested the dataset. The system
provides entertainment for users and useful datasets for researchers.

\section{Unit Testing/Functional Testing}
\subsection{PhotoHunter}
\subsection{Quick Pic}
\subsection{Researcher Interface}
 

\section{System Testing}
System testing will focus on the interaction between the PhotoHunter app, Quick Pic app, researcher interface, and the PhotoHunter API. 
The expected flow of the project is as follows:
\begin{itemize}
\item The PhotoHunter application should successfully insert new images with relevant information into the database.
\item The PhotoHunter API should serve images and relevant tags from the database to the Quick Pic application.
\item The Quick Pic application should successfully provide 
\end{itemize}
\section{Customer Testing}

\end{document}
