\documentclass{article}

\title{PhotoHunter: A Citizen-Scientist Game for Collecting Computer 
Vision Datasets}
\author{Connor Greenwell \and Ryan Baltenberger 
  \and J.\ David Smith \and Aaron Bradshaw}

\begin{document}

\maketitle

\section{What is ``Citizen-Science''?}

Citizen-science is a form of crowdsourcing in which participants
collect or analyze data. Typically, the tasks they perform are simple
in nature. For example, they may be asked to take a photo, measure the 
temperature, or translate text.

This model is successful due to the ability of humans to perform such
tasks extremely easily and in large numbers, often at little to no
cost in time or effort. For a single research team to collect such
large amounts of data on their own would be both costly and time
consuming. Often, automating these tasks is impossible for computers,
or is in fact the very aim of a projects principle investigators.

\section{Why a Game?}

people get bored when theres no payoff.

people like games. 

people like leaderboards (im number one! woo!)

treating photohunter as a game, and abstacting the citizen-science out
will keep people interested (longer)

\section{What Are ``Computer Vision Datasets''?}

As in other fields related to Artificial Intelligence or Machine
Learning, having data to train or test on is integral to the success
of Computer Vision projects. In Computer Vision (often referred to as 
``CV''), these datasets are collections of images or videos, each with
associated metadata. For example, in \cite{islam2014geofaces} they
propose a collection of front-facing facial images each with an
accompanying latitude-longitude pair. Similarly, in \cite{X} they
propose a collection of Y with assocated Z and W.

What both \cite{islam2014geofaces}, \cite{X}, an others have in common
is the difficulty they experienced when collecting these datasets. In
this document, and the following sections we will propose a system for
reducing these difficulties by proposing a system for crowd-sourcing
images and videos, along with an interface for asserting the validity
of the accompanying metadata.

\section{Saving Money}

Researchers have approached dataset creation and labeling by using 
Amazon Mechanical Turk in the past.  Mechanical Turk is a system where
jobs can be posted for users to complete.  The users are given a small 
amount of compensation for completing these tasks.  The downside to this
for researchers is the requirement to fund the completion of the jobs
that they post.  This costs the researchers money that could be used
for other parts of their project instead of dataset creation.  While jobs
can be posted for less than \$0.10, this can quickly add up when you have
tens and even hundreds of thousands of jobs to fund.

\section{PhotoHunter}

\subsection{Data Submission via Mobile App}

\subsection{Web Interface for Voting on Submissions}

\subsection{Web Interface for Requesting Datasets}

\end{document}
